\documentclass[12pt,a4paper]{report}
% Packages for enhanced functionality
\usepackage[utf8]{inputenc}
\usepackage[T1]{fontenc}
\usepackage{graphicx} % For including images
\usepackage{geometry} % For page layout
\usepackage{hyperref} % For clickable links and references
\usepackage{fancyhdr} % For custom headers and footers
\usepackage{titlesec} % For section title formatting
\usepackage{float} \usepackage{circuitikz}
\usepackage{caption}
\usepackage{siunitx}
\usepackage{amsmath}
\usepackage{subcaption}
\usepackage{booktabs}
\newcommand{\vecb}[1]{\mathbf{#1}}
\newcommand{\brak}[1]{\ensuremath{\left(#1\right)}}
\newcommand{\cbrak}[1]{\ensuremath{\left\{#1\right\}}}
\newcommand{\abs}[1]{\left\vert#1\right\vert}
\newcommand{\norm}[1]{\left\lVert#1\right\rVert}
\providecommand{\sbrak}[1]{\ensuremath{{}\left[#1\right]}}
\providecommand{\lsbrak}[1]{\ensuremath{{}\left[#1\right.}}
\providecommand{\rsbrak}[1]{\ensuremath{{}\left.#1\right]}}
\providecommand{\brak}[1]{\ensuremath{\left(#1\right)}}
\providecommand{\lbrak}[1]{\ensuremath{\left(#1\right.}}
\providecommand{\rbrak}[1]{\ensuremath{\left.#1\right)}}
\providecommand{\cbrak}[1]{\ensuremath{\left\{#1\right\}}}
\providecommand{\lcbrak}[1]{\ensuremath{\left\{#1\right.}}
\providecommand{\rcbrak}[1]{\ensuremath{\left.#1\right\}}}
\hypersetup{
    colorlinks=true,  % Enable colored text links
    linkcolor=orange,    % Internal links (sections, table of contents, etc.)
    urlcolor=orange,     % External URLs
    citecolor=orange,    % Citations
    pdfborder={0 0 0} % Remove ugly default borders
}
\begin{document}
\title{\textbf{Experiment 5}\\
\LARGE{\textbf{ }}
\author{Akshara Sarma (EE224BTECH11003 \\ Arjun Pavanje (EE24BTECH11005)}

\begin{center}
\end{center}
\vspace{30pt}
\begin{figure}[ht]
	\centering
	\includegraphics[width = 100pt]{logo.png}\\
\end{figure}
\begin{center}
	Bachelor of Technology\\
	\vspace{10pt}
	Department of Electrical Engineering\\
\end{center}
}
\maketitle
\section{Aim}
\begin{enumerate}
\item Measure the DC I–V characteristic of a diode and extract diode parameters (ideality factor $\mathbf{n}$ and saturation current $\mathbf{I_s}$). 
\item Measure the small-signal (dynamic) resistance $\mathbf{r_d}$ around a chosen bias point and verify the small-signal model. 
\item Further, bias the diode with a DC voltage and apply a small AC signal. Observe the output on the CRO and use FFT mode to see harmonics. The diode current is nonlinear,
\begin{align*}
I \approx I_0 + g_1v + \frac{1}{2}g_2*v^2.
\end{align*}
\begin{itemize}
\item For small $V_{ac}$, only the fundamental appears. 
\item As $V_{ac}$ increases, 2nd and higher harmonics appear, showing nonlinearity. 
\item Measure fundamental and 2nd harmonic amplitudes vs $V_{ac}$ and note slopes $~1$ and $~2$
\end{itemize}
\end{enumerate}
\section{I-V characteristics of Diode}
\section{Small Signal Analysis}
\section{FFT Harmonics}
\end{document}