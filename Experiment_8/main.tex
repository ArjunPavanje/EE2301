\documentclass[12pt,a4paper]{article}
% Packages for enhanced functionality
\usepackage[utf8]{inputenc}
\usepackage[T1]{fontenc}
\usepackage{graphicx} % For including images
\usepackage{geometry} % For page layout
\usepackage{hyperref} % For clickable links and references
\usepackage{fancyhdr} % For custom headers and footers
\usepackage{float} 
\usepackage{circuitikz}
\usepackage{caption}
\usepackage{graphicx}
\usepackage{subcaption}
\usepackage{hyperref}
\usepackage{siunitx}
\usepackage{tikz}
\usepackage{circuitikz}
\usepackage{amsmath}
\usepackage{tikz}
\usepackage{subcaption}
\usepackage{booktabs}
\newcommand{\vecb}[1]{\mathbf{#1}}
\newcommand{\brak}[1]{\ensuremath{\left(#1\right)}}
\newcommand{\cbrak}[1]{\ensuremath{\left\{#1\right\}}}
\newcommand{\abs}[1]{\left\vert#1\right\vert}
\newcommand{\norm}[1]{\left\lVert#1\right\rVert}
\providecommand{\sbrak}[1]{\ensuremath{{}\left[#1\right]}}
\providecommand{\lsbrak}[1]{\ensuremath{{}\left[#1\right.}}
\providecommand{\rsbrak}[1]{\ensuremath{{}\left.#1\right]}}
\providecommand{\brak}[1]{\ensuremath{\left(#1\right)}}
\providecommand{\lbrak}[1]{\ensuremath{\left(#1\right.}}
\providecommand{\rbrak}[1]{\ensuremath{\left.#1\right)}}
\providecommand{\cbrak}[1]{\ensuremath{\left\{#1\right\}}}
\providecommand{\lcbrak}[1]{\ensuremath{\left\{#1\right.}}
\providecommand{\rcbrak}[1]{\ensuremath{\left.#1\right\}}}
\hypersetup{
    colorlinks=true,  % Enable colored text links
    linkcolor=orange,    % Internal links (sections, table of contents, etc.)
    urlcolor=orange,     % External URLs
    citecolor=orange,    % Citations
    pdfborder={0 0 0} % Remove ugly default borders
}


\title{\textbf{Lab Report: Experiment 8}}
\author{EE24BTECH11003 : Akshara Sarma Chennubhatla\\EE24BTECH11005 : Arjun Pavanje}

\begin{document}
\maketitle
\begin{center}
	\textbf{Experiment:\\}Designing a CE amplifier\\cascaded with a CC amplifier\\for a gain of 40. Plotting the\\frequency response (Both Magnitude and phase)\\of the amplifier on log-log graph.
\end{center}
\vspace{30pt}
\begin{figure}[h!]
	\centering
	\includegraphics[width = 100pt]{logo.png}\\
\end{figure}
\begin{center}
	Bachelor of Technology\\
	\vspace{10pt}
	Department of Electrical Engineering\\
\end{center}
\newpage

\section{Theory}

\subsection{Common-Emitter (CE) Amplifier}
Common Emitter configuration can provide high gain, with inverted output (180 degree phase shifted). However, this does have issues related to stability, gain unpredictability (due to unintentional positive feedback). \newline 
Gain in Common Emitter is given by,
\begin{align*}
    A_v &= \dfrac{V_{in}}{V_{out}} = \dfrac{-g_mR_C}{g_mR_E+1}\\
    &\approx -\frac{R_C}{R_E}
\end{align*}
This is a reasonable approximation as $g_mR_E >> 1$. And hence, gain depends only on resistance values as long as BJT is in saturation.
\begin{figure}[H]
\centering
\resizebox{0.6\textwidth}{!}{
\begin{circuitikz}
\tikzstyle{every node}=[font=\normalsize]
\draw (5,10) to[Tnpn, transistors/scale=1.19] (5,12.5);
\draw (5,10) to[R,l={ \normalsize $R_E$}] (5,8.5);
\draw (5,12.5) to[R,l={ \normalsize $R_C$}] (5,14);
\draw (5,8.5) to (5,8.25) node[sground]{};
\draw (4,11.25) to[sinusoidal voltage source,l={ \normalsize $V_{in}$}] (1.25,11.25);
\draw (5,8.25) to[short] (1.25,8.25);
\draw (1.25,8.25) to[short] (1.25,11.25);
\draw (5,14.75) to[american voltage source, l={\normalsize $V_C$}] (8.75,14.75);
\draw (5,14) to[short] (5,14.75);
\draw (8.75,14.75) to[short] (8.75,8.25);
\draw (8.75,8.25) to[short] (5,8.25);
\draw (5,12.25) to[short, -o] (6.25,12.25) ;
\node [font=\normalsize] at (6.75,12.25) {$V_{out}$};
\end{circuitikz}
}
\caption{Common Emitter Circuit}
\end{figure}
\subsection{Common-Collector (CC, Emitter Follower) Amplifier}
Common collector amplfifier provides a gain of nearly unity. Output voltage follows input voltage so, it is also called \textbf{Emitter follower}. \newline
\begin{align*}
    A_v = \dfrac{V_{out}}{V_{in}} \approx 1
\end{align*}
\begin{figure}[H]
\centering
\resizebox{0.6\textwidth}{!}{
\begin{circuitikz}
\tikzstyle{every node}=[font=\normalsize]
\draw (5,10) to[Tnpn, transistors/scale=1.19] (5,12.5);
\draw (5,10) to[R,l={ \normalsize $R_E'$}] (5,8.5);
\draw (5,8.5) to (5,8.25) node[sground]{};
\draw (4,11.25) to[sinusoidal voltage source,l={\normalsize $V_{in}$}] (1.25,11.25);
\draw (5,8.25) to[short] (1.25,8.25);
\draw (1.25,8.25) to[short] (1.25,11.25);
\draw (5,12.5) to[american voltage source, l = {\normalsize $V_C$}] (8.75,12.5);
\draw (8.75,8.25) to[short] (5,8.25);
\draw (5,10.25) to[short, -o] (6.25,10.25) ;
\node [font=\normalsize] at (6.75,10.25) {$V_{out}$};
\draw (8.75,8.25) to[short] (8.75,12.5);
\end{circuitikz}
}
\caption{Common Collector Circuit}
\end{figure}

\subsection{Cascading CE and CC Stages}
\begin{figure}[H]
\centering
\resizebox{0.6\textwidth}{!}{%
\begin{circuitikz}
\tikzstyle{every node}=[font=\normalsize]
\draw (5,10) to[Tnpn, transistors/scale=1.19] (5,12.5);
\draw (5,10) to[R,l={ \normalsize $R_E$}] (5,8.5);
\draw (5,12.5) to[R,l={ \normalsize $R_C$}] (5,14);
\draw (5,8.5) to (5,8.25) node[sground]{};
\draw (4,11.25) to[sinusoidal voltage source,l={ \normalsize $V_{in}$}] (1.25,11.25);
\draw (5,8.25) to[short] (1.25,8.25);
\draw (1.25,8.25) to[short] (1.25,11.25);
\draw (5,14) to[short] (5,14.75);
\draw (5,12) to[short] (7.5,12);
\draw (9,10) to[Tnpn, transistors/scale=1.19] (9,12.5);
\draw (7.5,12) to[short] (7.5,11.25);
\draw (7.5,11.25) to[short] (8,11.25);
\draw (5,8.25) to[short] (9,8.25);
\draw (5,14.75) to[short] (9,14.75);
\draw (9,14.75) to[short] (9,12.5);
\draw (6.75,13.25) to (6.75,13) node[sground]{};
\draw (9,8.25) to[R,l={ \normalsize $R_E'$}] (9,10.25);
\draw (9,10.5) to[short, -o] (10,10.5) ;
\node [font=\normalsize] at (10.5,10.5) {$V_{out}$};
\draw (6.75,14.75) to[american voltage source,l={ \normalsize $V_{C}$}] (6.75,13.25);
\end{circuitikz}
}%
\caption{Cascaded circuit}
\end{figure}
\pagebreak
\section{Circuit}
\begin{figure}[h!]
    \centering
    \includegraphics[width=0.8\linewidth]{Experiment_8/figs/circuit.jpeg}
\end{figure}
In the circuit built, the values of resistors used are:
\begin{align*}
    R_C &= 82.6k\Omega\\
    R_E &= 1.5k\Omega\\
    R_E' &= 2k\Omega
\end{align*}
The voltage values used are:
\begin{align*}
    V_C &= 10V\\
    V_{in} &= V_{DC} + V_{AC\text{ small signal}}\\
    V_{DC} &= 0.630V\\
    V_{AC} &= 100mVpp
\end{align*}

The above voltage values are specifically experimentally determined and used considering the following:
\begin{itemize}
    \item For high values of $V_{DC}$, a higher $V_C$ is needed to keep the BJT in the saturation region. 
    \item If sufficient $V_C$ is not provided, then the gain will not reach 40 and will saturate near 1 as it is now linear with the $V_{DC}$ since it is in linear region.
    \item If $V_{DC}$ is too low, then the BJT will not provide 40 gain and will reach a max gain of < 1 as the BJT is not even on i.e., it won't let the current pass through the output terminal.
    \item So, a sweet spot in the middle is selected which allows gain of 40.
\end{itemize}
The above resistor values are determined considering the following:
\begin{itemize}
    \item The ratio of above resistor values is not exactly 40 but around 54. This is because actual gain is not $\dfrac{R_C}{R_E}$ but $\dfrac{R_C}{R_E + r_e}$ where $r_e = \dfrac{V_T}{I_e}$.
    \item So if ratio is exactly 40, then output gain will only be around 35-36.
    \item Also, since cascading is happening, the gain is further reduced to around 0.9-0.99 times the actual value of gain.
    \item So, to counter-balance all of this, ratio is taken to be around 54.
\end{itemize}

\section{Procedure}

\begin{enumerate}
    \item The experiment was conducted to study the frequency response of a cascaded Common Emitter (CE) and Common Collector (CC) amplifier designed for a midband voltage gain of approximately 40.
    
    \item The amplifier circuit was constructed and the input signal was applied using a function generator, while the output was observed on a dual-channel oscilloscope.
    
    \item For each selected frequency, both the input and output waveforms were displayed simultaneously on the oscilloscope screen.
    
    \item Using the oscilloscope cursors, the peak-to-peak voltages of the input (\(V_{\text{in,pp}}\)) and output (\(V_{\text{out,pp}}\)) signals were measured.
    
    \item The voltage gain at each frequency was calculated using the relation:
    \[
        A_v = \frac{V_{\text{out,pp}}}{V_{\text{in,pp}}}
    \]
    
    \item To determine the phase difference between the input and output signals, the horizontal time difference (\(\Delta x\)) between corresponding peaks of the two waveforms was measured.
    
    \item The phase shift was then calculated using the expression:
    \[
        \phi = 2 \pi f \Delta x
    \]
    where \(f\) is the applied signal frequency.
    
    \item The above measurements were repeated for a wide range of frequencies to obtain the magnitude and phase response of the amplifier.
    
    \item The variation of gain and phase with frequency was analyzed to verify the expected midband gain of approximately 40 and to study the amplifier’s bandwidth characteristics.
\end{enumerate}


\section{Experiment Images:}
\pagebreak

\begin{figure}[h!]
    \centering
    % Row 1
    \begin{subfigure}[b]{0.45\textwidth}
        \includegraphics[width=\textwidth]{Experiment_8/figs/1.jpeg}
    \end{subfigure}
    \hfill
    \begin{subfigure}[b]{0.45\textwidth}
        \includegraphics[width=\textwidth]{Experiment_8/figs/2.jpeg}
    \end{subfigure}

    % Row 2
    \begin{subfigure}[b]{0.45\textwidth}
        \includegraphics[width=\textwidth]{Experiment_8/figs/3.jpeg}
    \end{subfigure}
    \hfill
    \begin{subfigure}[b]{0.45\textwidth}
        \includegraphics[width=\textwidth]{Experiment_8/figs/4.jpeg}
    \end{subfigure}

    % Row 3
    \begin{subfigure}[b]{0.45\textwidth}
        \includegraphics[width=\textwidth]{Experiment_8/figs/5.jpeg}
    \end{subfigure}
    \hfill
    \begin{subfigure}[b]{0.45\textwidth}
        \includegraphics[width=\textwidth]{Experiment_8/figs/6.jpeg}
    \end{subfigure}

    % Row 4
    \begin{subfigure}[b]{0.45\textwidth}
        \includegraphics[width=\textwidth]{Experiment_8/figs/7.jpeg}
    \end{subfigure}
    \hfill
    \begin{subfigure}[b]{0.45\textwidth}
        \includegraphics[width=\textwidth]{Experiment_8/figs/8.jpeg}
    \end{subfigure}

\end{figure}

\pagebreak

\section{Values Obtained}

\begin{table}[htbp]
\centering
\caption{Frequency vs Magnitude and Phase}
\begin{tabular}{|c|c|c|}
\hline
\textbf{Frequency (Hz)} & \textbf{Magnitude (dB)} & \textbf{Phase (rad)} \\
\hline
$1$                 & $40.4$  & $3.142$ \\
$10$                & $42.4$  & $3.142$ \\
$25$                & $42.0$  & $3.078$ \\
$50$                & $42.4$  & $3.142$ \\
$100$               & $42.8$  & $3.142$ \\
$250$               & $42.8$  & $3.062$ \\
$500$               & $42.4$  & $3.142$ \\
$1.00 \times 10^3$  & $40.8$  & $3.142$ \\
$2.50 \times 10^3$  & $41.6$  & $3.142$ \\
$5.00 \times 10^3$  & $41.6$  & $3.015$ \\
$1.00 \times 10^4$  & $42.4$  & $3.015$ \\
$2.50 \times 10^4$  & $38.4$  & $2.748$ \\
$4.00 \times 10^4$  & $37.6$  & $2.714$ \\
$5.00 \times 10^4$  & $36.0$  & $2.670$ \\
$7.50 \times 10^4$  & $32.8$  & $2.450$ \\
$1.00 \times 10^5$  & $28.8$  & $2.138$ \\
$2.50 \times 10^5$  & $14.8$  & $1.885$ \\
$5.00 \times 10^5$  & $8.20$  & $1.131$ \\
$7.50 \times 10^5$  & $5.52$  & $0.659$ \\
$1.00 \times 10^6$  & $4.24$  & $0.503$ \\
$2.00 \times 10^6$  & $2.24$  & $0.377$ \\
$5.00 \times 10^6$  & $1.08$  & $0.063$ \\
\hline
\end{tabular}
\end{table}

\pagebreak

\section{Results}
\begin{figure}[h!]
    \centering
    \includegraphics[width=1\textwidth]{Experiment_8/figs/plot.png}
    \caption{Magnitude, Phase response}
\end{figure}
\begin{figure}[h!]
    \centering
    \includegraphics[width=0.9\textwidth]{Experiment_8/figs/spice.png}
    \caption{LT-Spice simulation}
\end{figure}
\section{Observations}

\begin{itemize}
    \item Gain lies between $41-42V$ in the constant region and it dies down afterward (from frequencies greater than $1kHz$). So bandwidth is $1kHz$.
    \item The CE stage sets the overall voltage gain and provides phase inversion of $180^\circ$.
    \item The CC (emitter follower) stage provides unity voltage gain.
    \item So final gain, is obtained by multiplying the 2 gains.
\end{itemize}

\subsection{Why cascade?}
\begin{enumerate}
    \item \textbf{Common Emitter}
    \begin{itemize}
        \item \textbf{Advantages}
        \begin{itemize}
            \item High gain
        \end{itemize}
        \item \textbf{Pitfalls} \newline
        \begin{itemize}
            \item High output impedance, and hence can't be used to drive low impedance loads (as it would lead to reduced gain, inefficient power transfer, etc).
            \item Limited cutoff frequency
        \end{itemize}
        \textbf{Note: } Output Impedance is defined as the voltage drop as observed from the load i.e. voltage drop when a load is connected.
    \end{itemize}
    \item \textbf{Common Collector}
    \begin{itemize}
        \item \textbf{Advantages}
        \begin{itemize}
            \item High input impedance, low output impedance. hence low impedance devices can comfortably be driven.
            \item Very large current gain
        \end{itemize}
        \item \textbf{Pitfalls}
        \begin{itemize}
            \item No voltage amplification
        \end{itemize}
    \end{itemize}
    
\end{enumerate}

\begin{itemize}
    \item Cascading ensures larger bandwidth, minimal distortion, noise, power efficiency.
    \item Cascading combines the high gain from CE configuration while maintaining low output impedence from CC configuration to provide the best of both worlds.
    \item Cascading ensures high gain, while ensuring amplified signal doesn't drop (i.e. eliminating voltage loss).
\end{itemize}

\section{Conclusion}
Cascaded CE, CC amplifier was built with a gain of 40. Phase and magnitude responses were plotted.
\end{document}
